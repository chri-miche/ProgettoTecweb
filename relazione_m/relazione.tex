\documentclass[12pt, a4paper]{article}

\usepackage[T1]{fontenc}
\usepackage[utf8]{inputenc}
\usepackage[italian]{babel}
\usepackage{graphicx}
\graphicspath{ {img/} }

\usepackage{multirow}

\begin{document}

\begin{titlepage}
\begin{center}
Indirizzo del sito: tecweb2021:8080/cmichele/ \\
Gli account utente usano una e-mail per effettuare l'accesso. Fanno eccezione le seguenti coppie: \\
user / user
admin / admin
\vfill
\textbf{Autori} \\
christian.micheletti@studenti.unipd.it \\
jacopo.fichera@studenti.unipd.it
\end{center}
\end{titlepage}

\newpage

\tableofcontents

\newpage

\section{Analisi}
\subsection{Utenza}
Sono previsti tre tipi di utenza, in base al livello di autorità: \textbf{utente non autenticato}, \textbf{utente autenticato} e \textbf{amministratore}. Le funzionalità sono associate alla classe di utenza a cui sono rivolte, e un livello di autorità ha accesso anche alle funzionalità dei livelli minori. In altre parole, le funzionalità pubbliche sono disponibili per ogni classe di utenza, le funzionalità dell'utente standard sono disponibili sia per questi ultimi sia per gli amministratori e le funzionalità amministrative sono esclusive degli amministratori.
\subsection{Funzionalità pubbliche}
\paragraph{Accesso pubblico.} Una sezione dell'applicativo e alcune funzionalità sono di accesso pubblico. L'accesso pubblico è attivo per default, quando ci si collega alla pagina \texttt{/home.php}. La pagina principale è composta da una sezione di ricerca e navigazione, una fascia dove si riportano le informazioni sulla propria posizione e il \textbf{feed}. Il feed è composto da tre modalità di ricerca: tra i più recenti, i più popolari e i più dibattuti.
\paragraph{Registrazione di un nuovo account.} Si effettua alla pagina \texttt{/register.php}. Occorre inserire il proprio nome utente, l'indirizzo e-mail e una password.
\paragraph{Ricerca post.} La funzionalità si utilizza tramite l'input di ricerca in alto (desktop) o nel menù laterale (mobile), scrivendo il testo da cercare, per poi premere "invio" oppure cliccare il tasto "ricerca".  
\paragraph{Visualizzazione di un post.} Che si sia arrivati dal feed o dai risultati della ricerca, ogni visitatore può visualizzare le informazioni del post: il titolo, il nome dell'utente che l'ha postato, i commenti e le immagini dell'avvistamento. Un visitatore non autenticato da qui può anche visitare il profilo degli utenti che hanno commentato o dell'utente che ha postato.
\paragraph{Catalogo.} Il catalogo permette di cercare delle informazioni sugli uccelli, suddivisi per ordine, famiglia e classe.
\subsection{Funzionalità per utenti}
\paragraph{Accesso all'area riservata.} 
Si effettua dalla pagina \texttt{/login.php}. Occorre inserire l'e-mail e la password scelte al momento della creazione dell'account. 
\paragraph{Funzionalità di interazione con il post.} Un utente autenticato può lasciare un commento, lasciare un like o un dislike sul post che sta leggendo. Il commento può essere formattato anche con tag <b> per ottenere delle parole in grassetto ed <em> per ottenerne in corsivo.
\paragraph{Modifica del profilo personale.}
Un utente può accedere al proprio profilo tramite il menù laterale, alla voce "Profilo". Dal proprio profilo è possibile modificare la propria immagine, premendo su "Scegli file" e poi su "Cambia immagine profilo".
\paragraph{Aggiungere degli amici.} È possibile accedere alla pagina di un altro utente premendo sul link di un post. Qualora si avesse effettuato l'accesso, apparirà il bottone "Aggiungi ai seguiti", se si vuole seguire un utente. Se l'utente era già nella lista dei seguiti, allora il bottone riporterà la scritta "Rimuovi dai seguiti". 
\paragraph{Creazione di un Post.} La funzionalità principale per l'utente autenticato è quella di creazione del post. Per creare un post occorre:
\begin{itemize}
\item inserire un titolo;
\item inserire un numero qualsiasi di foto;
\item inserire una descrizione.
\end{itemize}
Il titolo deve essere testo semplice, mentre la descrizione può contenere i tag <b> ed <em>, rispettivamente per grassetto e corsivo. Il numero di foto è variabile: bisogna però caricarle tutte insieme, non è possibile caricarne alcune per poi aggiungerne altre. È anche possibile non caricare neanche una foto, per porre domande semplici alla community.
\subsection{Funzionalità di amministrazione}
\paragraph{Gestione raw delle tabelle.}
Il pannello di amministrazione è accessibile solo ed esclusivamente agli utenti amministratori. Il pannello ha un elenco di voci, dalle quali è possibile visualizzare, inserire, modificare ed eliminare le seguenti entità:
\begin{itemize}
\item utente;
\item ordine;
\item famiglia;
\item genere;
\item ordine;
\item specie;
\item stato di conservazione.
\end{itemize}
Premendo su una voce si navigherà su un elenco di tutte le entità in banca dati. Ogni riga rappresenta le informazioni riguardo a quell'entità, inoltre avrà due bottoni, uno per effettuare una modifica, l'altro per effettuare l'eliminazione. Vicino al titolo è presente inoltre un bottone per la creazione di una nuova riga. Fa eccezione la tabella utenti: nella tabella utenti, un amministratore non può modificarne il nome utente, la password e nemmeno l'immagine. Un amministratore può anche eliminare un utente che tiene un comportamento negativo sul sito e non può creare utenze.
\section{Navigazione}

\section{Sviluppo}
\subsection{Database}
\subsection{Backend}
Il backend è stato sviluppato interamente in php 7.0.2. Non sono stati usati framework esterni, ma è stato implementato un framework liberamente ispirato alle tecnologie più recenti. 
\subsubsection{Il framework Component}
Il framework è organizzato secondo il paradigma ad oggetti. La classe principale è la classe astratta Component. Ogni componente erediterà le funzionalità di questa classe. La public interface di ogni component si può riassumere come segue:
\begin{itemize}
\item un costruttore che prende una stringa che corrisponde al contenuto da mostrare, con dei placeholder per i dati da sostituire;
\item un metodo \texttt{build} che sostituisce i placeholder, costruendo la pagina e restituendola;
\item un metodo \texttt{returnComponent} che costruisce la pagina se non era ancora stata costruita;
\item un metodo di conversione automatica in stringa, che corrisponde ad una chiamata \texttt{returnComponent()};
\item un metodo \texttt{notBuilt} che indica di ricostruire la pagina alla prossima \texttt{returnComponent()};
\item un metodo \texttt{resolveData} che restituisce un array associativo contenente i dati da sostituire nei placeholder. Deve essere sovrascritto nelle implementazioni delle classi figlie.
\end{itemize}
\subsubsection{Test}
\subsection{Interfaccia}
\subsubsection{Test}
\subsubsection{Registrazione}
\begin{center}
\begin{tabular}{|c|c|c|r|}
\hline
\textbf{Nome utente} & \textbf{Password} & \textbf{e-mail} & \textbf{Risultato} \\
\hline
Alfonso Cuaron & alcua & alfie@cuar.it & \textbf{Successo} \\
John Wick & wick & john@wick.gun & \textbf{Successo} \\
\hline
\end{tabular}
\end{center}


\end{document}